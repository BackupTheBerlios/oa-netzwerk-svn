\documentclass[a4paper,12pt]{scrreprt}
\usepackage{ngerman}
\usepackage[T1]{fontenc} % Schriftartfestlegung
\usepackage[utf8]{inputenc}
\usepackage{url}

% Title Page
\title{Installationsanleitung f"ur den OAN-Server}
\author{Michael K"uhn}

\begin{document}
\maketitle
\tableofcontents

\begin{abstract}
In diesem Dokument wird beschrieben, wie der Server f"ur OA-Netzwerk aufegestzt und eingerichtet wird.
\end{abstract}

\chapter{Einf"uhrung}
Willkommen! In diesem Kapitel finden Sie einen kurzen "Uberblick dar"uber was OAN ist, wo die Software zu beziehen ist und unter welchen Lizenzen Software und Dokumentationen ver"offentlicht sind.
\section{Was ist OAN}
Hier erscheint demn"achst eine kurze Einf"uhrung in OAN.
\section{Woher bekomme ich die Software}
Software gibts bei Berlios
\section{Aufbau dieses Dokuments}
3 Kapitel: 1. Einf"uhrung, 2. Systemvorraussetzungen, 3. Installation
\section{Lizenzen}
Dieses Dokument steht unter einer Creative Commons Namensnennung-Keine kommerzielle Nutzung-Weitergabe unter gleichen Bedingungen 3.0 Deutschland Lizenz.
\chapter{Systemvorraussetzungen}
\section{Hardware}
Getestet auf SPARC-Enterprise-T2000, sowie AMD Athlon(tm) 64.
Sollte auf allen Platformen, die Java unterst"utzen, laufen. Keine Gew"ahr.
\section{Betriebssysteme}
Getestet auf SunOS 5.10, sowie Debian 4.0. Sollte auf allen Betriebssystemen, die Java unterst"utzen laufen. Keine Gew"ahr.
\section{Software}
Ben"otigt Java mind. Version 1.6, Apache Tomcat 6.0\\
Apache Commons Bilbiotheken:\\
beanutils 1.7.0\\
cli 1.1\\
codec 1.3\\
collections 3.2\\
dbcp 1.2.2\\
digester 1.8\\
httpclient 3.1\\
lang 2.3\\
logging 1.1\\
pool 1.3\\
sowie die Bilbliotheken:
catalina \\
jconn3\\
jdom 1.1\\
junit 4.4 (nur f"ur tests)\\
log4j 1.2.15\\
naming-factory-dbcp\\
naming-factory\\
naming-resources\\
servlet-api\\
\\
sowie zum Kompilieren und Erstellen der Pakete Ant Version 1.7
\chapter{Installation}
\section{Vorbereitungen}
Zuerst mu"s ein Benutzer f"ur den Servlet Container angelegt werden. Danach den Servlet-Container installieren (mit Tomcat 6.0 getestet).
Datenbank mit allen Tabellen unter Sybase anlegen.
Sobald dies geschehen ist, log4j.properties in \begin{verbatim}
TOMCAT_HOME/conf 
\end{verbatim} ablegen.
Dann db-users.conf anlegen.
\begin{verbatim}commons-codec, commons-httpclient, commons-logging, jdom\end{verbatim} und log4j in \begin{verbatim}TOMCAT_HOME/lib\end{verbatim} ablegen.
\section{Konfigurieren}
Here m"ussen alle Konfigurationssachen aufgeschrieben werden.
\section{Kompilieren}
mit Hilfe von Ant oadminrealm.jar (ant jarOAdminRealm) und resourcerealm.jar (ant jarResourceRealm) erzeugen und in \begin{verbatim}TOMCAT_HOME/lib\end{verbatim} kopieren. Danach ant warServer aufrufen und restserver.war erzeugen. Entweder per tomcat-manager deployen oder nach \begin{verbatim}TOMCAT_HOME/webapps\end{verbatim} kopieren.
\chapter{Anhang}
\section{ConfigFiles}
\end{document}
